\documentclass[12pt]{article}
\usepackage{graphicx} % Required for inserting images
\usepackage[style=apa]{biblatex} % Feel free to change the style to another one you prefer
    \addbibresource{MACS-3200-Wang-Cosmo.bib}
\usepackage{geometry}
    \geometry{margin=1in}
\usepackage{times} % Use Times New Roman font

% Set font size to 12pt
\renewcommand{\normalsize}{\fontsize{12}{15}\selectfont}

% Set double spacing using linespread
\linespread{2.0} % Adjust as needed for your double spacing requirement

\title{A network analysis to the governance structure of the semiconductor supply chain }
\author{Cosmo Wang}
\date{\today}

\begin{document}

\maketitle

\textbf{Link to my Github Repo:} \href{https://github.com/Cosmo280/organizational-network-analysis}
\\
\section*{\centering Introduction}
\textbf{}
\indent In 2020, industries using semiconductors as raw materials began to experience a pandemic-induced chip shortage. This is majorly the result of automobile maker’s decision to cut down their orders for chips in the beginning of COVID outbreak due to their expectation of decline in market demand for cars. When the automobile market began to rebound, the chipmakers had already reallocated their manufacturing capacity for other customers, making it hard for car-related chips to be acquired (Miller, 2022).\\ 
\indent In response to the shortage, the transaction cost branch of the neoclassical economic paradigm predicted that organizations would internalize the production process to reduce transaction cost (Williamson, 1981; 1991; 1993; 2010). However, that is not entirely the case. Although automotive-related integrated device manufacturers (IDM), which are corporations that design, manufacturer, and sell microchips, have invested in building new foundries or reopen shuttered fabs to ensure control over their production (Piovaccari, 2022; Texas Instruments, 2023; Renesas, 2024), IDMs, as well as fabless chip designers, both in and beyond the automotive industry, continue to rely on outsourcing for chip supply and deepen their partnership with the few key actors in semiconductor manufacturing (TSMC et al., 2023; GlobalFoundries 2021). Moreover, public and private actors, triggered by the fear of future shortages, have expressed a heightened urgency for state intervention in global supply chain security and participate actively in the state policy making process to address their interests (Miller 2022). Several policies were thus created in aim to subsidize the few leading-firms in the oligopoly market of semiconductor manufacturing (SIA 2022; SAI, 2024; European Commission, 2022; Shivakumar, Wessner, and Howell, 2023), keeping the organizational boundaries blurry.\\ 
\indent The shortage exposed the complexity of vertical and lateral coordination between firms of different industries in the semiconductor supply chain. With more industries becoming dependent on it and chip supply being a strategic resource in the post-Covid era, mapping the governance structure of the semiconductor supply chain became prominent to the study, management, and security of global supply chains. What the governance structure of the semiconductor supply chain looks like, however, could not be easily answered relying only on the theory of transaction cost without the supplements of other organizational theories, supply chain studies, and social network analysis(SNA). To depict the governance structure of a multi-industry production network on a global scale, this paper proposes an organizational approach to map the network structure of interconnected firms among different industries nested in a global production network. To do so, I would first review the theories and topologies of supply chains to lay the foundation for examining the semiconductor industry. Then, with transaction cost economics(TCE) and its analysis of economic organizations, I could further perform the microanalytics of vertical and lateral integration of the organizations embedded in the supply chain. Lastly, by visiting organizational theories on embedding and the existing works on elite network analysis, I would conceptualize organizational embeddedness and propose an approach to operationalize cohesion clusters in the supply chain as the measurement of the concept. With the supply chain topology, organizational boundaries, and cohesion clusters combined, I would argue that the inter-industry semiconductor supply chain in the network form consists of cohesive clusters located in the center with core organizational actors tightly connected within and the rest of the organizational actors distributing in the periphery, loosely connected to actors in these center clusters on market relationships.

\section*{\centering Literature Review}
    \subsection*{\centering Chain governance and network dynamics of global supply chains} 
\textbf{}
\indent Global value chain(GVC) and Global production network(GPN) are two major explanatory frameworks for understanding the structure of a deeply interdependent global economy (Neilson et al., 2014; Kano et al., 2020). The concept of value chain was first introduced by Michael Porter in 1985, where he defined value chain as a series of activities where a firm performs to create a marketable product. The term supply chain is often used interchangeably with a similar definition that removes the value adding sequences. It is widely recognized that Gereffi’s work in 1994 expanded the concept to a global scale and provided the first explanatory framework on the power relationship between large buyers and their producers (Neison et al., 2014; Kano et al., 2020). Gereffi (1994; also 2018) introduced the framework buyer-driven global commodity chain (GCC), in which he operationalized the measurement of organizational participation in the global economy on three dimensions: 1.) an input-output structure, which is a series of connected products and services in a sequence of value-adding economic activities, 2.) a territoriality, which is the spatial dispersion or concentration of enterprises in the production and distribution networks, and 3.) a governance structure, which is the authority and power relationship that determines the flow direction of information, knowledge, and resources between organizational actors (Gereffi, 1994; also Gereffi, 2018, p.44-45). However, despite its popularity for subsequent empirical studies, the GCC framework was criticized for not being a theory but rather a methodology, creating a need for theorization of global chain analysis.\\
\indent Building on the inter-firm governance dimension of the original GCC framework, Gereffi et al. came up with the most significant theorization of GVC governance in 2005 (also 2018). Gereffi et al. examined three intersecting supply-chain variables: 1.) complexity of transaction, 2.) codifiability of transactions, and 3.) the capabilities within the supply-base to identify a fivefold topology of governance structure within GVCs (2005). Ranging the level of explicit coordination and power asymmetry from low to high, the five topologies are: market, modular, relational, captive, and hierarchy. Yeung and Coe(2015) in their work on GNP 2.0 further argues that Gereffi et al. 's governance topology goes beyond firm relationships in the same industry and applies to inter-industry relationships between firms with partnership on a global scale. “Relational” and “modular” chain governance were more likely than others to appear when the strategic partners and specialized suppliers who lead firms formed partnerships with are providing information technology services.\\
\indent Based on the GVC-GNP theories, the industrial governance structure of the semiconductor industry is most likely to be a modular chain. In the 1990s, major American and European firms in electronics made the decision to outsource their manufacturing departments. A significant portion of their production capacity was outsourced to a few contracted manufacturers on a global scale. These global contracted manufacturers, or “turn key suppliers” in Gereffi et al’s terminology, which are firms that fabricate asset specific products for buyers and its production capacity is flexible to be transferred to other customers, modularized the value chain activities for the wide variety of lead firms to access, introducing the high modularity to the governance of the semiconductor global supply chain (Gereffi et al, 2005). Both the GVC and GNP theorists noticed that modular governance is playing an increasingly important role in the global economy as the design and manufacturing components of cutting-edge electronics became more separated due to immense pressure of competition. By forming inter-firm partnerships between the lead firms and their specialized supplier, firm-level specialization in the production of electronics has been proven to be particularly effective (Gereffi et al, 2005; Yeung and Coe, 2015). 

    \subsection*{\centering Transaction cost economics and embedded networks}
\textbf{}
\indent After examining the semiconductor supply chain from a macro perspective, we now turn to neoclassical economics and organizational theories to have a better understanding of its microstructure. Since the GVC-GNP theories were laid on the foundation of transaction cost economics(Gereffi et al, 2005; Yeung and Coe, 2015), it would be a good starting point.\\
\indent Transaction cost economics provided an economic approach to analyze when activities would occur within an organization and, by doing so, the governance structure of the inter-firm relationships(Williamson, 1981; 1991; 1993). Williamson(1981) argues that when asset is nonspecific, which buyers could turn to alternative sources and suppliers could sell their output to other buyers, firms tend to rely on commonly owned site-specific assets for finishing the production processes to reduce transaction cost; when production capacity as an asset has become more specific, transaction cost increases and organizations were more likely to internalize the asset production. This would give the organization complete access over all relevant information when settling disputes. He further argues in his 1991 work in an industry with strong property rights, the make-or-buy question would eventually end up with either market, hierarchy, or hybrid as the governance structure. He believes that the hybrid structure is an intermediate state with joint ventures as safeguarding against betrayal between firms and eventually these long-term contractual relationships would dissolve when one party learns enough to let go the other (Williamson, 1993).\\
\indent Growing on the concept of hybrid governance structure, however, the GVC theory disagrees with the conclusion that interconnected firms in a hybrid governance production system would always end up in vertical integration. The reasons for disagreement were two: First, learning to develop certain capabilities has become too difficult, time-consuming, and inefficient that even the most vertically integrated firm rarely internalizes all the production technology it requires. The entry barrier specific value chain activities, especially those in electronics, were so high that firms have no choice but to rely on external suppliers(Gereffi et al, 2005); secondly, network theorists (Coleman, 1988; Podolny and Page, 1998; Granovetter, 1985; Jarillo, 1988; Thorelli, 1986) argues that trust, reputation, and interdependence would suppress the opportunistic behavior of organizations, making long-term interdependence relationships between possible. With all the organizational theories above combined, Gereffi et al proposed that three governance structures of inter-firm relationships with high asset specificity should be market, hierarchy, and network. The network structure is later to be further specified into modular, relational, and captive governance topology of the GVC theory. Therefore, in retrospect, the modular supply chain could be conceptualized and operationalized as a dense network of social relations that is neither market nor hierarchy (Granovetter, 1985), with the big buyers and their specialized suppliers of assets with high specificity being the central core actors of the network.
}

\section*{\centering Methods and Data}
\textbf{}
\indent Drawing from theories of elite network analysis(Laumann, Knoke, and Kim, 1985; also Lauman and Knoke, 1987; Knoke, 1993), the first step to operationalize network analysis is to specify the system boundaries of data collection. To do so, a precise definition of what subject of analysis should be included. Based on the current configuration of the semiconductor industry(Thadani and Allen, 2023), the system boundaries, with organizations as units of analysis, would be specified as the following: 1.) big buyers of carchips in the automotive industry, fabless chip designers, chip-related intellectual property designers, and electronic design automation software designers are to be included in the design domain; specialized microchip manufacturers(both IDMs and pure plays), chip making equipment producers, as well as providers of chip manufacturing related raw materials and chemicals are to be included in the manufacturer domain; lastly, organizations specializing in testing chips and their assembly with other components of the marketable final product are to be included in the test and assembly domain. \\ 
\indent With the boundaries clarified, the next step is to select the core actors in each domain. A combination of positional and relational methods would be applied to generate the final list of elites. I would start with going through publicly accessible databases of government and business reports to identify the most influential and active organizational actors in each domain with content analysis using machine learning models. With the preliminary list of names generated, I would then apply a snowball sampling technique where I select additional actors to the list based on the supplier/buyer relations of the existing elites until a satisfied amount of organizational actors is reached. Data on supplier/buyer relations would be collected from Refinitive(previous Thompson ONE), a database with records on supplier/buyer relations. Further data collection from SEC files and news coverage to improve accuracy. \\
\indent In my exploratory data analysis(EDA), based on the distribution of microchip application market share structure(Asthana et al, 2006), I broke the major buyers of microchips into the following six categories: 1.) computer equipment, 2.) consumer electronics, 3.) telecommunications, 4.) industrial equipment, 5.) aerospace and defense, and 6.) automotive industry. I scraped a total of 160 distinct firms from these six categories plus the semiconductor and semiconductor industry. These 160 organizations are my core actors. Then, using the names of these core actors, I transfer their name into their Refinitiv identifier and download the data of their supplier and buyers with a confidence score of information accuracy higher than 0.5. Those suppliers and buyers whose name is non-existent in the existing list of organizations would be added to the list as peripheral actors, until a total amount of 1500 firms were satisfied. \\
\indent With the list of core actors finalized, it’s time to decide how to operationalize organizational ties. Directorate interlockings, information flows, joint ventures, and supplier/buyer relations are four proxies of network ties in organizational elite studies (Mizruchi, 1992; Knoke, 1993; Carnorale and Yeniyurt, 2013). Networks of interlocked directorates are a common form of inter-organizational structure and it is perceived as a method to understand inter-firm relationships(Mizruchi, 1992; Moody and White, 2003). Nevertheless, due to the international nature of the semiconductor industry with different states having a variety of regulations on interlocking, interlocked board members as proxy for strong ties would not be the best choice. Similarly, although information exchange is used as a measurement of influence and power relations between network actors, it would more likely to be a proxy for weak ties for being inherently relational(Knoke, 1993; Granovetter, 1992). Joint ventures between firms within our system boundaries is a good choice to measure structural embeddedness for its lack of directionality and its common use for inter-industry partnerships(Carnovale and Yeniyurt, 2013). However, it is noticeable that the increasing importance of semiconductors is a relatively new thing, with the US-China trade war in 2018 and COVID being breaking points of increased demand. My initial exploratory data analysis on the existing data of joint ventures among organizational actors may have shown that relying on joint ventures as proxy for organizational ties do not provide results with high significance. Therefore, I would argue that supplier/buyer relationships would play the best proxy for strong ties due to its high quantity and accessibility. A series of non-repeated and non directional tuples indicating supplier/buyer relationships is extracted from the snowball sampling data used to locate organizational actors in the previous step. \\
\indent Last but not least, the only thing left is to decide the measurement of network embeddedness. Moody and White(2003) provided a method of operationalization. With structural cohesion defined as “The minimum number of actors who, if removed from a group, would disconnect the group(Moody and White, 2003, P. 103)”. Based on this definition, by blockmodeling(or group in simple terms) the actors with higher cohesion, the more tightly connected subgroups in a larger network structure could be discovered, with one actor belonging to multiple subgroups and larger groups simultaneously. The more groups an actor belongs to, the more the actor is embedded in the network. This methodology provides an approach that depicts the connectivity, the clustering, and the core-periphery relations among network actors and thus would be a workable choice to map the network form of governance structure in the semiconductor industry. This is done by first simulating a small world model of the larger network to have a better understanding of the scaled up structure. The small world model could represent the cohesive clusters in the center of the supply chain network.


\section*{\centering Preliminary Work and Discussion}
\textbf{}
\indent I first performed a small world simulation of the network structure based on a shrunken dataset by randomly selecting 30 core actors and 60 peripheral actors from my list of organizational actors and the existing supplier/buyer relationships between them. With Python packages of Networkx for network analysis and Matplotlib, I was able to produced the following visualization: \\
\indent With the core actors marked in red and the peripheral actors in blue, it is demonstrated in a clear fashion that most core actors were embedded in a cluster with high cohesion in the center of the network while almost all peripheral actors were distributed in the peripheral and loosely connected to a core actor playing a bridge to structure holes. \\
\indent Then on a larger scale, by using the original dataset, the following graph was produced: \\
\indent Due to the increasing network complexity and the loss of details when representing three dimensional space with two dimensional visualization, the scaled up network of the supply chain is not as intuitive as the small world simulation on the analysis of governance structure. However, I would consider this visualizational to be successful for it continues to provide support for my argument of the network form of supply chain governance structure in the semiconductor industry. \\

\section{\centering Proposed Timeline and Feasibility Assessment}
\textbf{
\indent
}


\end{document}